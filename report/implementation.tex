\section{Implementation}\seclabel{impl}

We briefly describe our implementation of the DAA protocol on the IBM~4758
PCI-X Cryptographic Coprocessor.
We developed (1) a
full implementation of DAA---issuer, verifier, and cryptocard algorithms---in the
functional programming language~OCaml and (2) a separate implementation of the cryptocard-specific algorithms in~C, a language we can readily compile
for execution on cryptocards.

We decided to divide our efforts because the potential benefits seemed to outweigh
the obvious cost.
First, our ML implementation is easier to
scrutinize, being relatively high-level and thus ``close'' to the protocol specification.
It served as the protocol reference when we started writing C code.
As important, several categories of bugs (\eg, memory safety errors) are easily avoided in ML;
we sought to avoid, as much as possible, routine debugging chores related to verifier/issuer code.
Second, our goal was to perform an attestation.
We felt that using separate code bases
would likely rule out some ``false positives'' caused
by bugs common to both.

\paragraph*{Status:}
Our ML implementation is complete, but bugs remain.
We know of at least one, wherein the code to verify an issuer's parameters
reject parameters that should be well-formed.

Our C implementation is incomplete, lacking some portions of
the DAA signing protocol. It has not been debugged, either in isolation
or in combination with our ML implementation. We have
not performed an attestation involving both code bases.

Our C implementation relies on the GMP library
to perform big integer operations that are not supported by the
cryptocard; for example, linear arithmetic. However, some functionalities that are
implemented now as utility functions need to be reimplemented to use similar
functionalities offered by the cryptocard.

All of our code is available electronically at
\url{https://github.com/elnikety/crypsys}.
