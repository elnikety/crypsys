\section{Introduction}

\if 0
\begin{comment}
	\emph{PDS:}
	Our goal is to motivate and sketch out our work.
	
	At the moment, I've simply dropped in a copy of our proposal.
	Even if those ideas suffice, we clearly need to adjust the text since
	we're now talking about what we have done rather
	than what we propose to do.
	
	Let's consider improving the ideas.
	
	I want to move away from
	TCG-style attestation to the more general problem of
	\emph{authenticating a remote application}.
	Identifying
	all of the software running on a machine may well uniquely
	identify the machine.
	Authentication without inherently leaking identity enables
	attestation. What other protocols does it enable?
	Voting protocols?
\end{comment}
\fi

We implement Direct Anonymous Attestation (DAA) on top of
a member of the IBM~4758 family of
cryptographic coprocessors.

Attestation is a core concept in so-called trusted computing.
It enables trust decisions based on the identity of a machine
and its software stack.
Why identity?
Consider the following scenario.
Suppose server $S$ houses content $C$ and wants to
only ever release $C$ to a trusted application $A$;
for example, $S$ might supply music online,
$C$ might comprise songs,
and $A$ might be a trusted application that
lets its users listen to---but not copy---their purchases.
Certainly, $S$ would employ cryptographic protocols to protect
its content during download, but what about \emph{after} download?
How does $S$ ensure it's really talking to $A$? Moreover, how does
it ensure that no other software running alongside $A$ makes an illicit
copy of $C$?
Attestation attempts to mitigate such concerns; for example,
prior to sending $C$, the server $S$ might identify all of the
software running alongside $A$ to ensure that only ``trustworthy'' programs
are running.

As this scenario suggests, we are concerned with \emph{remote} attestation.
Remote attestation involves authenticating a trusted component (e.g., a trusted platform module~(TPM))
on a remote machine prior to running an attestation protocol
with it.
Authentication matters: $S$ would like to know that when it performs a remote attestation,
its peer is \emph{honest}.
By communicating with a TPM, $S$ avoids trusting
the rest of a user's machine.
But authentication poses a problem: A TPM's identity serves as
\emph{personally identifying information.}
In our scenario, for example, if remote attestation reveals a TPM's identity, then
online companies might misuse that information to track a user and profile
her preferences.

The DAA authenticates a TPM
while offering anonymity guarantees.
DAA performs the necessary authentication \emph{without} revealing the TPM's identity,
sending $S$ a zero-knowledge proof instead. \dchk{Similarly, DAA can be used to authenticate a machine that uses a configuration out of a set of trusted configurations without revealing the signature of that configuration. This is useful when the configuration conveys too much information and, thus, could be used as an identifier.}

In this project, we adapt DAA to work with an IBM secure coprocessor
rather than a TPM.
Such coprocessors serve as dedicated machines, 
designed to run arbitrary application-specific
code in a controlled, secure environment.
The TPM specification takes a far less
interesting and (in the long-term) less maintainable approach to
secure coprocessing: One size fits all.
More important, the IBM coprocessors have been verified to FIPS~140-1 Level~4.
Among other things, this means they offer laboratory-tested physical security against
environmental attacks.
Keys and other sensitive data stored in an IBM coprocessor's battery-backed RAM
are automatically `zeroized' when such attacks are detected.
To our knowledge, no TPM has been so verified.

The contributions of this project to the trusted computing community are two fold:

(1) Provide cryptocard-specific DAA implementation -- to the best of our knowledge, there is no out-of-the-box implementation. This should enable trusted computing systems to harness the numerous hardware features provided by cryptocards to deliver richer systems. As we discuss in Section~\ref{sec:impl}, we provide implementations in ML and C.

(2) Less complex implementation;  due to the hardware limitations of the TPM, traditional DAA implementations offload expensive computations to the relatively more powerful processor of the host machine. However, results of offloaded computations have to be verified by the TPM chip since the host is not trusted. In our model, we eliminate the distinction between the host and the cryptocard. 

\if 0
We expect this project to challenge us.
We will describe our adapted DAA protocol and implement it.
Time permitting, we will prove its security properties.
Our implementation may or may not involve instrumenting (some subset of)
a BIOS, a bootloader, and an OS to take appropriate \emph{code measurements} for attestation;
if not, we will provide a well-documented API so that others,
more familiar with a particular system, can do so.

\begin{comment}
	\emph{PDS and EE:}
	These are the contributions we hoped for a few days ago.

	They should be the points covered by our draft---not the story though.
	\begin{itemize}
	\item What can a cryptocard, compared to TPM, add to the picture?
	\item Why DAA is useful?
	\item There is no out-of-the-box DAA implementation for cryptocards.
	\item Cryptocard-specific DAA eases the implementation since we don't have
to offload any computations to the host.
	\item We have a *working* implementation in ML
	\item We have a *working* implementation in C for the cryptocard
	\end{itemize}
\end{comment}
\fi