\section{Protocol overview}\seclabel{daa}

In our setting, the DAA protocol involves three participants:
\begin{description}

\item[\textbf{Issuer}]
An issuer runs a setup procedure to establish
parameters for itself and other participants.
The issuer publishes these parameters along with a
\emph{zero-knowledge proof} that the parameters
are well-formed. (We discuss such proofs in~\secref{zkp}. Intuitively,
they relate secret values to public values, without leaking the secrets.)

Other protocol participants check an issuer's parameters
once, amortizing that cost over all subsequent uses of those parameters.
In the following, we presuppose that other protocol participants have
checked an issuer's parameters.

\item[\textbf{Cryptocard}]
After mutual authentication, a cryptocard and an issuer run the \emph{join protocol}
to create a \emph{certificate} for later use by the cryptocard.
The certificate serves as evidence that the issuer has authenticated the cryptocard.

The join protocol employs blinding factors and two zero-knowledge proofs
to (a) prevent the issuer from learning the secret key material, $f$, used by the
cryptocard for DAA and (b) convince the issuer that the rogue tag
sent by the cryptocard relates to $f$.

A cryptocard's \emph{rogue tag} is a value calculated from an issuer's DAA parameters,
the cryptocard's secret $f$, and the identity of its communication peer.
Rogue tags serve as pseudonyms that permit a limited
form of tracking sufficient for issuers and verifiers (q.v.) to heuristically
detect misbehaving cryptocards.

\item[\textbf{Verifier}]
A verifier and a cryptocard run the \emph{signing protocol}.
In it, the cryptocard uses a zero-knowledge proof to convince
the verifier that (a) a certificate over $f$ exists and thus the cryptocard
is authentic and (b) the rogue tag sent by the cryptocard relates to $f$.

\end{description}
Brickell et al.\ describe a fourth protocol participant, the untrusted host
housing a TPM.
They do so in order to offload some computations from a resource-starved TPM
to its host.
As cryptocards are more powerful than TPMs, we avoid the distinction.
In our implementation, a cryptocard performs all TPM and
host calculations.\footnote{%
	We do not avoid computations relevant to the Host-TPM divide.}

\bigskip

Before describing the underlying cryptographic operations,
we offer a bird's-eye view of the join and signing protocols.

\paragraph*{Join protocol}
The join protocol presupposes that the issuer and the cryptocard
have authenticated one another and established a secure
channel. The issuer knows it's talking to a legitimate
cryptocard and vice versa. Messages arriving at the card are
authentic. The protocol proceeds as follows.
\[
	\minCDarrowwidth200pt
	\begin{CD}
	\text{Cryptocard}	@>{\displaystyle U,\, N_I}>>	\text{Issuer} \\
	{}	@<{\displaystyle n_i}<<	{} \\
	{}	@>{\displaystyle P,\, n_h}>>	{} \\
	{}	@<{\displaystyle P',\, C,\, v''}<<	{}
	\end{CD}
\]
The cryptocard initiates the protocol, sending a blinded
signature request $U$ and rogue tag $N_I$.
The request $U$ is built up from the issuer's parameters,
the cryptocard's secret $f$, and
a fresh blinding factor $v'$.
Upon receiving the issuer's nonce $n_i$, the cryptocard
sends a zero-knowledge proof $P$ that
$U$ and $N_I$ were well-formed along with a nonce $n_h$.
The cryptocard's proof mentions $n_i$.
Upon verifying $P$ and confirming that $N_I$ isn't blacklisted, the
issuer sends an unblinding factor $v''$, a certificate $C$ over
$f$ and the quantity $v \defeq v'+v''$, and a zero-knowledge proof
$P'$ that $C$ is well-formed.
The issuer's proof mentions $n_i$ and $n_h$.

\paragraph*{Signing protocol}
The signing protocol is much simpler:
\[
	\minCDarrowwidth200pt
	\begin{CD}
	{}	@>{\displaystyle m,\, n_v,\, b}>>	\text{Cryptocard} \\
	\text{Verifier}	@<{\displaystyle \sigma,\, N_V}<<	{} \\
	\end{CD}
\]
The cryptocard receives a message $m$ to sign along
with a nonce $n_v$ and
a bit $b$ indicating whether $m$ originated with the verifier or within the cryptocard.
Hidden inputs include the relevant issuer's parameters and an optional
basename for the verifier.
The cryptocard responds with its rogue tag $N_V$ for this verifier
and a ``signature'' $\sigma$ comprising
a zero-knowledge proof that (a) the cryptocard knows a certificate $C$
from the issuer and (b) both $C$ and $N_V$ relate to the cryptocard's secret $f$.
The proof mentions $m$, $n_v$, and $b$.
