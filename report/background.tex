\section{Crypographic building blocks}

We offer an informal account of the building blocks used by DAA.
DAA comprises a group signature scheme~(\secref{cl}) paired with a rogue tagging scheme~(\secref{rogue}), using \emph{efficient} zero-knowledge proofs to achieve anonymity~(\secref{zkp}).

DAA's need for both computational and communication efficiency should be clear:
Devices with limited resources communicate remotely to participate in DAA.
Thus, a constraint: One may design DAA-like protocols around any group
signature and rogue tagging schemes for which one can define suitably efficient
proof protocols.

\subsection{Group signature scheme}\seclabel{cl}

The Camenisch and Lysyanskaya~(CL) signature scheme
supports group creation,
the ability for a group's creator to add a member to the group,
and the ability for a group member to prove membership to a third party.
The group of interest is the set of cryptocards that have proved
their identity to a DAA issuer.

The CL signature scheme
involves three operations built up over the group $\QR_n$
of quadratic residues for a special RSA modulus $n$:
\begin{itemize}

\item \textbf{Setup:} To create a new CL key,
one picks a special RSA modulus $n$, one of its prime
factors $p$,
and members $a_1$, \ldots, $a_L$, $b$, and $c$ of $\QR_n$
where $L$ is a parameter.
All but $p$ are public.

\item \textbf{Sign:}
A signature for a block of messages $m_1$, \ldots, $m_L$
and a key as above comprises a random prime $e$, a random $s$, and a
value $v$ satisfying
\begin{equation}\label{cl/signat}
	v^e \equiv a_1^{m_1} \cdots a_L^{m_L} b^s \pmod n.
\end{equation}

\item \textbf{Verify:}
To verify a signature $(e, s, v)$ over a message block $m_1$, \ldots, $m_L$,
one need only compute both sides of \eqref{cl/signat} and check for equality:
$$	
	v^e \bmod n \iseq a_1^{m_1} \cdots a_L^{m_L} b^s \bmod n.
$$
\end{itemize}
A signer---knowing the secret $p$ and thus how to factor $n$---can
easily compute $v$ satisfying \eqref{cl/signat}
whereas any algorithm to forge such a signature
reduces to an algorithm to solve the flexible RSA problem.

For important restrictions (\ie, parameters
governing the lengths of $n$, the $m_i$, $e$, and
$s$) and a proof of unforgeability by reduction to the flexible RSA problem, please
see~\cite{cl}.

\subsection{Rogue tagging scheme}\seclabel{rogue}

The rogue tagging scheme defines pseudonyms for cryptocards and offers
a limited form of linkability.
The goal is to permit an issuer or verifier to apply heuristics to detect ``rogue'' cryptocards
and add such cryptocards to a blacklist.

To our knowledge, Brickell et al.\ introduced DAA's rogue tagging scheme
when they defined DAA.
That's rather unfortunate for an outsider attempting to understand the scheme in isolation:
The rogue tagging scheme and its security properties are only implicitly defined as a part of the larger (and much more complicated) DAA protocol.

During setup, an issuer picks a group $G$ such that the discrete
logarithm problem in $G$ has about the same difficulty as
factoring CL RSA moduli.
Many issuers and verifiers can share the same rogue-tagging
group. DAA assigns distinct tags to issuers/verifiers via
so-called basenames. As far as the protocol is concerned,
basenames are uninterpreted bit strings; see \S\/6.2 in~\cite{smyth}
for a critique and practical advice.

While running the join protocol with an issuer or the signing
protocol with a verifier that supplies its basename, a
cryptocard computes a base $\zeta \in G$ from its peer's name and
commits to the rogue tag $N \defeq \zeta^f$, where $f$ is the cryptocard's
secret key for use in DAA signatures. (When signing with a
verifier that does not supply a basename, the cryptocard picks
a random $\zeta \in G$.)

\medskip

DAA's rogue tagging key comprises primes $\Gamma$ and $\rho$ and a member $\gamma$
of the multiplicative group mod $\Gamma$ such that $\gen\gamma$ is a subgroup of order $\rho$ and $\rho$ is a
large prime factor of $\Gamma - 1$.
All of $\Gamma$, $\rho$, and $\gamma$ are public.

There are at least two important points to make regarding the length of $\rho$:
\begin{itemize}

\item It's the maximum bit length of cryptocard secrets $f$.

\item It must be chosen large enough to make it difficult to
compute discrete logarithms in $\gen\gamma$; see the discussion around
Algorithm 4.84, ``Selecting a $k$-bit prime $p$ and a generator $\alpha$
of $\Zstar p$'' in~\cite{handbook}.

\end{itemize}

We offer no “check if this tag is a rogue” heuristic. In
theory, issuers and verifiers use tags to track legitimate
users (pseudonymously) and to identify rogues. We offer tags,
but no heuristics. Each issuer/verifier must settle on a
policy. (The DAA paper suggests that when talking to a
platform with tag $N$, a verifier checks $N$ and any earlier $N'$
used by that platform against a blacklist. If found, the
verifier should abort the protocol.)

\subsection{Zero-knowledge proofs}\seclabel{zkp}

\if 0
\begin{comment}
	\emph{PDS:}
	Security property:  Unforgeability (by reduction to the
	flexible RSA problem).

	Alleged security properties: Satistical witness
	indistinguishable and soundness (by reduction to the flexible
	RSA problem in the random oracle model).
\end{comment}
\fi

DAA's zero-knowledge proofs ensure that all identifying information
about a cryptocard rests in the rogue tags it supplies to its protocol peers.

The DAA protocol is interesting, in part, because it is a practical
and widely-deployed protocol that relies heavily on zero-knowledge
proofs of knowledge.
We introduce such proofs using an extended example: The
proof of knowledge of a discrete logarithms modulo a
composite.\footnote{%
	This is a special case of one of the zero-knowledge proofs used in DAA;
	see~\cite[Appendix~A]{daa:full}.}

\subsubsection{The discrete log problem}

Let the group $G$ and $g$, $h \in G$ be given.
Suppose a prover $P$ and
verifier $V$ agree on these and $P$ wants to convince $V$ that
\begin{align*}
\exists \alpha.\, \alpha = \log_h g &\iff \exists \alpha.\, h^\alpha = g \\
	&\iff g \in \gen h.
\end{align*}
without revealing the witness $\alpha$.

\subsubsection{Making the verifier happy: Basic protocol}

Consider the following protocol. It has the form ``commit,
challenge, response'' (a so-called $\Sigma$-protocol).
\[
	\minCDarrowwidth200pt
	\begin{CD}
	P	@>{\displaystyle t \defeq h^r	\quad\text{for $r \ge 1$ fresh}}>>	V \\
	P	@<{\displaystyle b	\quad\text{for $b \in \sset{0, 1}$ fresh}}<<	V \\
	P	@>{\displaystyle s \defeq r - b\alpha}>>	V
	\end{CD}
\]
After running the protocol, the verifier knows $t$, $b$, and $s$
and so can check
$$
	t \iseq g^b h^s.
$$
From the verifier's perspective, the protocol is great. The
point:
\begin{align*}
	t = g^b h^s &\iff h^r = g^b h^{r - b\alpha} \\
	&\iff r = b \log_h g + r - b\alpha \\
	&\iff b\alpha = b \log_h g \\
	&\iff b = 0 \lor (b = 1 \land \alpha = \log_h g).
\end{align*}
Thus a prover cannot answer both challenges $b = 0$ and $b = 1$
correctly without knowing $\log_h g$. Intuitively, the protocol
satisfies \emph{soundness}: If a (potentially adversarial) prover
survives $k$ runs of the protocol, then with probability $2^{-k}$,
the prover knows $\log_h g$.

\subsubsection{Making the prover happy: An RSA group}

Here's another intuition: A $\Sigma$-protocol satisfies \emph{witness
indistinguishability} if there exists a simulator for the
prover such that (a) the simulator doesn't know the prover's
secrets and yet (b) the distribution of ``conversations''
between the verifier and the prover matches the distribution
of conversations between the verifier and the simulator.

If we could restrict our parameters $G$, $g$, and $h$ in order to
deny a (potentially adversarial) verifier any ``computational
back doors'' to the prover's secret $\alpha$, then we might attempt to
prove our protocol satisfies witness
indistinguishability.

We restrict our parameters as follows.
Let $n$ be a special RSA modulus so that
there exist primes $p$, $p'$, $q$, and $q'$ satisfying
$n = pq$, $p = 2p' + 1$, $q = 2q' + 1$, and $q \not= p$.
Let $G$ be the multiplicative group mod $n$
and $h$ a random generator of the group of quadratic
residues mod $n$.
Thus, $\size{\QR_n} = p'q'$. Our protocol specializes to
\[
	\minCDarrowwidth200pt
	\begin{CD}
	P	@>{\displaystyle t \defeq h^r \bmod n	\quad\text{for $r \in \sset{1, \ldots, p'q'}$ fresh}}>>	V \\
	P	@<{\displaystyle b	\quad\text{for $b \in \sset{0, 1}$ fresh}}<<	V \\
	P	@>{\displaystyle s \defeq (r - b\alpha) \bmod p'q'}>>	V
	\end{CD}
\]
where the verifier computes
\[
	v \defeq g^b h^s \bmod n
\]
and checks
\[
	t \iseq v.
\]

Now our imaginary proofs can argue that no such ``computational
back doors'' exists by reduction to the strong RSA assumption.

\subsubsection{Making the protocol practical: Avoiding communication}

We now introduce the so-called Fiat-Shamir heuristic, used to avoid
communication between the prover and the verifier.
The idea is to use a hash function as a source of challenge bits.

Let a hash function $H$ with output length $\ell_H$ be given.
The prover picks $\ell_H$ random values; computes the corresponding
commitments $t_1$, \ldots, $t_{\ell_H}$;
hits the common inputs and the commitments with $H$;
then computes $\ell_H$ responses using bits from
the hash as challenges. In practical terms, the prover hopes
that no adversary can defeat $H$ to discover
computational back doors to $\alpha$. Now our imaginary proofs must
make the random oracle assumption, in addition to the strong
RSA assumption.

\subsubsection{On efficiency}

The zero-knowledge protocol discussed in the preceding sections has one
drawback: Relying as it does on a list of $\ell_H$ commitments, it's not
very efficient.

The CL signature scheme and DAA use this inefficient protocol,
but only to establish that long-lived parameters (\eg, the CL group
parameters) are well-formed.
Each protocol participant checks such parameters once, amortizing that cost.

CL and DAA use many other zero-knowledge protocols to prove knowledge
of relations among discrete logarithms (among other relations).
These zero-knowledge protocols
are much more efficient, presupposing that parameters have been
checked, they ``get away with'' one commitment.
An account of these protocols is beyond the scope of this report;
please refer to~\cite[\S\/5]{cl} and~\cite[\S\/3.2]{daa} for references.
